\documentclass[a4paper,oneside,openany]{tufte-book}
\usepackage[utf8x]{inputenc}
\usepackage[german]{babel}
\usepackage{microtype} % Improves character and word spacing
\usepackage{booktabs} % Better horizontal rules in tables
\usepackage{ucs}
\usepackage{amsmath}
\usepackage{amsfonts}
\usepackage{amssymb}
\usepackage{graphicx}
\usepackage{color}
\usepackage{listings}
\usepackage{caption}

\makeatletter% since we're using commands with @ in their name

\let\origappendix\appendix% save a copy of the original meaning of \appendix
\renewcommand{\appendix}{%
  \origappendix% do all the original \appendix stuff
  \titlecontents{chapter}%
    [0em] % distance from left margin
    {\vspace{1.5\baselineskip}\begin{fullwidth}\LARGE\rmfamily\itshape} % above (global formatting of entry)
    {\hspace*{0em}\appendixname~\thecontentslabel: } % before w/label (label = ``2'')
    {\hspace*{0em}} % before w/o label
    {\rmfamily\upshape\qquad\thecontentspage} % filler + page (leaders and page num)
    [\end{fullwidth}] % after
  \titleformat{\chapter}%
    [display]% shape
    {\relax\ifthenelse{\NOT\boolean{@tufte@symmetric}}{\begin{fullwidth}}{}}% format applied to label+text
    {\itshape\huge Anhang~\thechapter}% label
    {0pt}% horizontal separation between label and title body
    {\huge\rmfamily\itshape}% before the title body
    [\ifthenelse{\NOT\boolean{@tufte@symmetric}}{\end{fullwidth}}{}]% after the title body
  \setcounter{secnumdepth}{0}% ``number'' the appendices
  \renewcommand{\thefigure}{\@arabic\c@figure}% define \thefigure to use only the figure number (1), not A.1
  \renewcommand{\thetable}{\@arabic\c@table}%
  %
  % Add any other special appendix-related code here.
  %
}
\makeatother% restore the special meaning of @


\author{Schett Matthias}
\title{SEN-\"{U}bung 3}

\begin{document}

\definecolor{gray}{rgb}{0.9,0.9,0.9} % background color for the listings

\lstset{language=[Visual]Basic, morekeywords={param, local}, breaklines=true, tabsize=4, showstringspaces=false,backgroundcolor=\color{gray}, numbers=left,
    basicstyle=\ttfamily} % standard listing settings

\frontmatter

\maketitle
\tableofcontents
\mainmatter

\chapter{Aufgabe 1}

\section{L\"{o}sungsidee}

Die Object Klasse dient nur als Basisklasse und verfügt nur einen Konstruktor der leer ist und über einen rein virtuellen Destruktor\sidenote{Dadurch wird Objekt zu einer abstrakten Klasse gemacht.}.
Die Klasse GraphicObject erbt public von Object und verfügt über 4 virtuelle Methoden, 2 davon sind wieder rein virtuell\sidenote{Dadurch wird auch GraphicObject zu einer abstrakten Klasse}.

\begin{itemize}
	\item void draw(); ist rein virtuell muss also in einer abgeleiteten Klasse überschrieben werden.
	\item TBox getBoundingBox() ist ebenfalls rein virtuell.\sidenote{Rein virtuell, da die Basisklasse nicht wissen kann, wie abgeleitete Klasse gezeichnet werden oder ihren Rahmen ermitteln.}
	\item getPositionHelper() const; wird benutzt falls eine abgeleitete Klasse ihre Implementierung für das Ermitteln der Position verändern will - ist protected
	\item moveToHelper(TCoord const\& position); gilt das gleiche wie bei getPositonHelper();
\end{itemize}

Weiters gibt es moveTo(TCoord const\& position) und getPositon(); diese beiden Methoden sind inline deklariert und rufen nur die Helper Methoden auf. Dadurch wird gewährleistet, dass es immer eine Standardimplementierung gibt
und Kunde nicht vergessen, die virutellen Methoden zu überschreiben.

In den abgeleiteten Klassen gibt es nur Implementierugen für die virutellen Methoden draw() und getBoundingBox(), die restlichen Methoden werden übernommen.

Die Klasse Picture wiederum ist ein Spezialfall. Sie verfügt über einen std::vector als Member der GraphicObject* aufnehmen kann, dadurch kann sie alles aufnehmen dass von GraphicObject ableitet, also auch weitere Pictures.
Die weiteren Funktionen rufen in Schleifen, die jeweilige Methode der im std::vector befindlichen Objekte auf.

\section{Testf\"{a}lle}
\lstinputlisting[caption=Testausgabe]{GraphicObject/OutputA1.txt}

\backmatter

\appendix

\setboolean{@mainmatter}{true}
\chapter{Aufgabe 1}\label{chap:Auf1}

\begin{fullwidth}
\lstinputlisting[caption=Typdefs]{GraphicObject/Types.h}
\lstinputlisting[caption=Object Header]{GraphicObject/Object.h}
\lstinputlisting[caption=GraphicObject Header]{GraphicObject/GraphicObject.h}
\lstinputlisting[caption=GraphicObject Implementierung]{GraphicObject/GraphicObject.cpp}
\lstinputlisting[caption=Line Header]{GraphicObject/Line.h}
\lstinputlisting[caption=Line Implementierung]{GraphicObject/Line.cpp}
\lstinputlisting[caption=Rectangle Header]{GraphicObject/Rectangle.h}
\lstinputlisting[caption=Rectangle Implementierung]{GraphicObject/Rectangle.cpp}
\lstinputlisting[caption=Circle Header]{GraphicObject/Circle.h}
\lstinputlisting[caption=Circle Implementierung]{GraphicObject/Circle.cpp}
\lstinputlisting[caption=Picture Header]{GraphicObject/Picture.h}
\lstinputlisting[caption=Picture Implementierung]{GraphicObject/Picture.cpp}
\lstinputlisting[caption=Testtreiber]{GraphicObject/Main.cpp}
\end{fullwidth}

\end{document}